\begin{center}
    \Huge \underline{\textbf{Sujet}}
\end{center}
Certains algorithmes de chiffrements à clef publique, comme le système RSA utilisent de grands nombres premiers. Nous verrons dans le cours d’arithmétique et cryptographie un test efficace permettant de savoir avec une grande certitude si un grand nombre est premier ou non. Cependant pour générer un grand nombre premier aléatoire ce test doit être appliqué un grand nombre de fois surtout si le nombre premier cherché doit avoir des propriétés particulières. Ceci rend la méthode coûteuse en temps et aussi en quantité d’aléas nécessaire. \\ \\
Il existe de nombreuses solutions visant à diminuer ce temps de calcul quitte à sacrifier un peu le caractère aléatoire des nombres premiers générés. D’autres solutions visent à produire des nombres premiers de manière proche de l’uniforme en diminuant la quantité d’aléas nécessaire.\\ \\
L’objectif du projet sera de comprendre et d’implémenter certaines de ces méthodes et de comparer les différents résultats expérimentaux vis à vis du temps de calcul, de la quantité d’aléas et du caractère aléatoire de la sortie.
